\documentclass[a4paper]{article}
\usepackage{ucs}
\usepackage[utf8x]{inputenc}
\usepackage[T1]{fontenc}
\usepackage[english]{babel}
\title{P4 notes}
\author{Andrius Dalinkevi\v cius}
\usepackage{float}
\usepackage{rotating}
\usepackage[pdftex,bookmarks=TRUE]{hyperref}
\usepackage{float}
\usepackage{rotating}
\usepackage{amsmath}
\usepackage{amssymb}
\usepackage{theorem}
\usepackage{calc}
\usepackage{graphicx}
\usepackage{lmodern}
\usepackage{bm}
\oddsidemargin=0cm
\topmargin=0cm
\headsep=0pt
\headheight=0pt
\textwidth=440pt
\textheight=670pt
\footskip=40pt

\begin{document}

\maketitle

This is the notes of the P4 task.

\section{Data}
There are two raw files with industrial production data:
'PIlocal.xlsx' and 'PIUSD.xlsx'. The first one contains observations
expressed in local currency, the second one - USD dollars: historic
fixed 2011 exchange rate and year on year exchange rate. Variations:
total, per capita and per household.

\subsection{Preparations}

\begin{enumerate}
\item Encoding. 'Country', 'Category' and 'Subcategory' columns were
  encoded. The ids and original names are saved in 'Country.id.csv'
  and 'Category.id.csv'. The script for this: '00encoding.R'.

\item Processing.
  \begin{itemize}
    \item Removal of unnecessary information(columns).
    \item Removal of rows which have all NA over the sample period.
    \item Removal of rows which have all zeros over the sample period.
    \item Making growth rates.
    \item Replacing Inf with NA.
  \end{itemize}

  Two scripts for data processing:

\item '01readata.R'. Objects:
  \begin{itemize}
    \item 'data.flow' - USD millions, year-on-year exchange rate.
    \item 'data.fix' - USD millions, fixed 2011 exchange rate.
    \item 'data.PC.flow' - USD per capita, year-on-year exchange rate.
    \item 'data.PC.fix' - USD per capita, fixed 2011 exchange rate.
    \item 'data.PHH.flow' - USD per household, year-on-year exchange rate.
    \item 'data.PHH.fix' - USD per household, fixed 2011 exchange rate.
  \end{itemize}

\item '02readata.R'. Objects:
  \begin{itemize}
    \item 'data.loc' - local currency millions.
    \item 'data.PC' - local currency per capita.
    \item 'data.PHH' - local currency per household.
  \end{itemize}
\end{enumerate}

\section{Testing}
\begin{itemize}
\item Initial data base. The script '30initDB.R' creates initial data
  base, 'P4.sqlite', with two tables: one for countries, other for
  categories. Each table contains two columns with ids and original
  names.
\item Testing the code for the insertion of correlations into
  database. Each code version was tested on 10000 insertions,
  sequentially. The time expressed in seconds:
  \begin{itemize}
    \item[-] Using a data.frame in the cycle vs without it: 126 vs
      94.
    \item[-] The Kendall function vs the cor function: 94 vs 84.
  \end{itemize}
The the code version using the cor function and inserting without
creating a data.frame works faster so far. The code for this is in '40time.R'.

\end{itemize}

\subsection{Histograms}

\begin{figure}[H]
\begin{center}
\includegraphics{"Entire_data_frame"}
\caption{All}
\label{fig:2}
\end{center}
\end{figure}


\begin{figure}[H]
\begin{center}
\includegraphics{"Japan"}
\caption{One country}
\label{fig:2}
\end{center}
\end{figure}

\begin{figure}[H]
\begin{center}
\includegraphics{"Industry"}
\caption{One industry}
\label{fig:2}
\end{center}
\end{figure}

\begin{figure}[H]
\begin{center}
\includegraphics{"JapanSample"}
\caption{One Country, sample: 1000, 3000, 5000, 20000, 50000}
\label{fig:2}
\end{center}
\end{figure}

\subsection{Correlation matrices, corrplot, within country}
In this subsection within country correlations are inspected. The data
is of the hierarchy level 4 and 3 countries were selected randomly. It
includes:

\begin{itemize}
\item Australia
\item Canada
\item India
\end{itemize}

The 'corrplot' package was used for sorting and visualizing
correlation matrices. The output of the four sorting methods are
depicted, for more details about sorting methods see ?corrMatOrder:

\begin{itemize}
\item "AOE" -  The angular order of the eigenvectors.
\item "FPC" - The first principal component order.
\item "hclust.ward" - for the hierarchical clustering order, ward.
\item "hclust.centroid" - for the hierarchical clustering order, centroid.
\end{itemize}

\subsubsection{AOE}

\begin{figure}[H]
\begin{center}
\includegraphics{"CorrMatrix/Australia_AOE"}
\caption{AOE method, Australia, categories of the hierarchy level 4 on x and y axes}
\label{fig:2}
\end{center}
\end{figure}

\begin{figure}[H]
\begin{center}
\includegraphics{"CorrMatrix/Canada_AOE"}
\caption{AOE method, Canada, categories of the hierarchy level 4 on x and y axes}
\label{fig:2}
\end{center}
\end{figure}

\begin{figure}[H]
\begin{center}
\includegraphics{"CorrMatrix/India_AOE"}
\caption{AOE method, India, categories of the hierarchy level 4 on x and y axes}
\label{fig:2}
\end{center}
\end{figure}

\subsubsection{FPC}

\begin{figure}[H]
\begin{center}
\includegraphics{"CorrMatrix/Australia_FPC"}
\caption{FPC method, Australia, categories of the hierarchy level 4 on x and y axes}
\label{fig:2}
\end{center}
\end{figure}

\begin{figure}[H]
\begin{center}
\includegraphics{"CorrMatrix/Canada_FPC"}
\caption{FPC method, Canada, categories of the hierarchy level 4 on x and y axes}
\label{fig:2}
\end{center}
\end{figure}

\begin{figure}[H]
\begin{center}
\includegraphics{"CorrMatrix/India_FPC"}
\caption{FPC method, India, categories of the hierarchy level 4 on x and y axes}
\label{fig:2}
\end{center}
\end{figure}

\subsubsection{hclust.ward}

\begin{figure}[H]
\begin{center}
\includegraphics{"CorrMatrix/Australia_hclust_ward"}
\caption{hclust.ward method, Australia, categories of the hierarchy level 4 on x and y axes}
\label{fig:2}
\end{center}
\end{figure}

\begin{figure}[H]
\begin{center}
\includegraphics{"CorrMatrix/Canada_hclust_ward"}
\caption{hclust.ward method, Canada, categories of the hierarchy level 4 on x and y axes}
\label{fig:2}
\end{center}
\end{figure}

\begin{figure}[H]
\begin{center}
\includegraphics{"CorrMatrix/India_hclust_ward"}
\caption{hclust.ward method, India, categories of the hierarchy level 4 on x and y axes}
\label{fig:2}
\end{center}
\end{figure}

\subsubsection{hclust.centroid}

\begin{figure}[H]
\begin{center}
\includegraphics{"CorrMatrix/Australia_hclust_centroid"}
\caption{hclust.centroid method, Australia, categories of the hierarchy level 4 on x and y axes}
\label{fig:2}
\end{center}
\end{figure}

\begin{figure}[H]
\begin{center}
\includegraphics{"CorrMatrix/Canada_hclust_centroid"}
\caption{hclust.ward method, Canada, categories of the hierarchy level 4 on x and y axes}
\label{fig:2}
\end{center}
\end{figure}

\begin{figure}[H]
\begin{center}
\includegraphics{"CorrMatrix/India_hclust_centroid"}
\caption{hclust.ward method, India, categories of the hierarchy level 4 on x and y axes}
\label{fig:2}
\end{center}
\end{figure}

\subsection{Correlation matrices,  RBGL,  cross countries}
The output of this subsection was deleted because it was useless. See
'issue.pdf'. The problem is still open.
% In this subsection cross country correlation are inspected. The
% hierarchy level of a data is 4. In order to choose
% country pairs, correlations of the category "Industrial (Entire
% Economy)" were depicted below. Names of rows and columns represent
% country and category, for example "17 1"  means country - 17 and
% category - 1. Three country pairs were chosen:

% \begin{itemize}
% \item Germany(6) - UK(17).
% \item Germany(6) - France(5).
% \item Germany(6) - Italy(9).
% \end{itemize}

% \begin{figure}[H]
% \begin{center}
% \includegraphics{"CorrMatrix/All"}
% \caption{Industrial (Entire Economy)}
% \label{fig:2}
% \end{center}
% \end{figure}


% The RBGL package was used for sorting. There are three functions available:
% \begin{itemize}
% \item cuthill.mckee.ordering.
% \item minDegreeOrdering.
% \item sloan.ordering.
% \end{itemize}

% Graphical results were very 'similar' so only output from the
% cuthill.mckee.ordering function are presented.

% \begin{figure}[H]
% \begin{center}
% \includegraphics{"CorrMatrix/Germany_UK_cut"}
% \caption{Germany vs UK}
% \label{fig:2}
% \end{center}
% \end{figure}

% \begin{figure}[H]
% \begin{center}
% \includegraphics{"CorrMatrix/Germany_France_cut"}
% \caption{Germany vs France}
% \label{fig:2}
% \end{center}
% \end{figure}

% \begin{figure}[H]
% \begin{center}
% \includegraphics{"CorrMatrix/Germany_Italy_cut"}
% \caption{Germany vs Italy}
% \label{fig:2}
% \end{center}
% \end{figure}









\end{document}



